\newglossaryentry{heig-vd}{
    name=HEIG-VD,
    description={Haute École d'Ingénierie et de Gestion du canton de Vaud}
}
\newglossaryentry{hes-so}{
    name=HES-SO,
    description={Haute École Supérieure de Suisse Occidentale}
}
\newglossaryentry{latex}{
    name=latex,
    description={Un langage et un système de composition de documents}
}
\newglossaryentry{maths}{
    name=mathematics,
    description={Les mathematiques sont ce que les mathématiciens fonts}
}
\newglossaryentry{os}{
    name=OS,
    description={Operating System}
}
\newglossaryentry{docker}{
    name=Docker,
    description={Docker est un outil permettant de gérer des containers, une sorte de machine virtuelle plus légère ayant pour but d'encapsuler une ou plusieurs applications / outils technologiques ainsi que toutes les dépendances que ces dernières exigent pour leur bon fonctionnement}
}
\newglossaryentry{kubernetes}{
    name=Kubernetes,
    description={Kubernetes est un système open-source permettant d'automatiser le déploiement, la mise à l'échelle et la gestion des applications conteneurisées}
}
\newglossaryentry{devops}{
    name=DevOps,
    description={Défini tout le processus conteant l'intégration continue, le déploiement continu et la livraison continue}
}
\newglossaryentry{ci}{
    name=CI,
    description={Continuous Integration / Intégration continue}
}
\newglossaryentry{cd}{
    name=CD,
    description={Continuous Deployement / Déploiement continu}
}

\newglossaryentry{github}{
    name=Github,
    description={GitHub is a website and cloud-based service that helps developers store and manage their code, as well as track and control changes to their code}
}

\newglossaryentry{gitlab}{
    name=Gitlab,
    description={GitLab is The DevOps Platform, delivered as a single application. This makes GitLab unique and creates a streamlined software workflow, unlocking your organization from the constraints of a pieced together toolchain}
}

\newglossaryentry{sgbd}{
    name=SGBD,
    description={Système de Gestion de Base de Données (DBMS) } %rajouter
}