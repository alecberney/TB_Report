\newglossaryentry{heig-vd}{
    name=HEIG-VD,
    description={Haute École d'Ingénierie et de Gestion du canton de Vaud}
}
\newglossaryentry{hes-so}{
    name=HES-SO,
    description={Haute École Supérieure de Suisse Occidentale}
}
\newglossaryentry{latex}{
    name=latex,
    description={Un langage et un système de composition de documents}
}
\newglossaryentry{docker}{
    name=Docker,
    description={Docker est un outil permettant de gérer des containers, une sorte de machine virtuelle plus légère ayant pour but d'encapsuler une ou plusieurs applications / outils technologiques ainsi que toutes les dépendances que ces dernières exigent pour leur bon fonctionnement}
}
\newglossaryentry{kubernetes}{
    name=Kubernetes,
    description={Kubernetes est un système open-source permettant d'automatiser le déploiement, la mise à l'échelle et la gestion des applications conteneurisées}
}
\newglossaryentry{devops}{
    name=DevOps,
    description={Défini tout le processus conteant l'intégration continue, le déploiement continu et la livraison continue}
}
\newglossaryentry{ci}{
    name=CI,
    description={Continuous Integration / Intégration continue}
}
\newglossaryentry{cd}{
    name=CD,
    description={Continuous Deployement / Déploiement continu}
}
\newglossaryentry{github}{
    name=Github,
    description={GitHub is a website and cloud-based service that helps developers store and manage their code, as well as track and control changes to their code}
}
\newglossaryentry{gitlab}{
    name=Gitlab,
    description={GitLab is The DevOps Platform, delivered as a single application. This makes GitLab unique and creates a streamlined software workflow, unlocking your organization from the constraints of a pieced together toolchain}
}
\newglossaryentry{sgbd}{
    name=SGBD,
    description={Système de Gestion de Base de Données (DBMS), logiciel système servant à stocker, à manipuler ou gérer, et à partager des données dans une base de données}
}
\newglossaryentry{mysql}{
    name=MySQL,
    description={Système de gestion de bases de données relationnelles (SGBDR)}
}
\newglossaryentry{fablab}{
    name=Fablab,
    description={Laboratoire permettant tout type de travaux situé à la HEIG-VD }
}
\newglossaryentry{teams}{
    name=Microsoft Teams,
    description={Plateforme collaborative de visioconférence appartenant à Microsoft}
}
\newglossaryentry{backend}{
    name=Backend,
    description={Terme désignant un étage de sortie d'un logiciel devant produire un résultat. On l'oppose au front-end (aussi appelé un frontal) qui lui est la partie visible de l'iceberg}
}
\newglossaryentry{frontend}{
    name=Frontend,
    description={Partie frontale du projet affichant l'interface utilisateur permettant d'utiliser la logique du backen}
}
\newglossaryentry{bd}{
    name=BD,
    description={Une base de données permet de stocker et de retrouver des données structurées, semi-structurées ou des données brutes ou de l'information, souvent en rapport avec un thème ou une activité. Celles-ci peuvent être de natures différentes et plus ou moins reliées entre elles}
}
\newglossaryentry{rest}{
    name=REST,
    description={REST (representational state transfer) est un style d'architecture logicielle définissant un ensemble de contraintes à utiliser pour créer des services web}
}
\newglossaryentry{api}{
    name=API,
    description={Une interface de programmation d’applications ou interface de programmation applicative (souvent désignée par le terme API pour Application Programming Interface) est un ensemble normalisé de classes, de méthodes, de fonctions et de constantes qui sert de façade par laquelle un logiciel offre des services à d'autres logiciels}
}
\newglossaryentry{framework}{
    name=Framework,
    description={En programmation informatique, un framework est un ensemble cohérent de composants logiciels structurels qui sert à créer les fondations ainsi que les grandes lignes de tout ou partie d'un logiciel, c'est-à-dire une architecture}
}
\newglossaryentry{php}{
    name=PHP,
    description={Language de programmation web}
}
\newglossaryentry{laravel}{
    name=Laravel,
    description={Framework web utilisant le language de programmation php}
}
\newglossaryentry{template}{
    name=Template,
    description={Modèle de projet web frontend définissant déjà toute l'interface graphique et étant adaptable au besoin}
}
\newglossaryentry{javascript}{
    name=JavaScript,
    description={Language de programmation web}
}
\newglossaryentry{vuejs}{
    name=Vue.js,
    description={Framework web frontend utilisant le language de programmation Javascript}
}
\newglossaryentry{nodejs}{
    name=Node.js,
    description={Environnement d’exécution JavaScript construit sur le moteur JavaScript V8}
}
\newglossaryentry{os}{
    name=OS,
    description={En informatique, un système d'exploitation (souvent appelé OS — de l'anglais Operating System) est un ensemble de programmes qui dirige l'utilisation des ressources d'un ordinateur par des logiciels applicatifs}
}
\newglossaryentry{conteneur}{
    name=conteneur,
    description={Un conteneur est une sorte de machine virtuelle allégée contenant une application}
}
\newglossaryentry{tcp}{
    name=TCP,
    description={Transmission Control Protocol, abrégé TCP, est un protocole de transport fiable, en mode connecté, documenté dans la RFC 7931 de l’IETF}
}
\newglossaryentry{git}{
    name=Git,
    description={Logiciel de gestion de versions décentralisé}
}
\newglossaryentry{apache}{
    name=Apache,
    description={Logiciel libre utilisé pour réaliser des serveurs HTTP}
}
\newglossaryentry{orm}{
    name=ORM,
    description={Un mapping objet-relationnel (en anglais object-relational mapping ou ORM) est un type de programme informatique qui se place en interface entre un programme applicatif et une base de données relationnelle pour simuler une base de données orientée objet}
}
\newglossaryentry{json}{
    name=JSON,
    description={JavaScript Object Notation (JSON) est un format de données textuelles dérivé de la notation des objets du langage JavaScript}
}
\newglossaryentry{http}{
    name=HTTP,
    description={L’Hypertext Transfer Protocol, généralement abrégé HTTP, littéralement « protocole de transfert hypertexte », est un protocole de communication client-serveur développé pour le World Wide Web}
}
\newglossaryentry{jwt}{
    name=JWT,
    description={JSON Web Token (JWT) est un standard ouvert défini dans la RFC 75191. Il permet l'échange sécurisé de jetons (tokens) entre plusieurs parties. Cette sécurité de l’échange se traduit par la vérification de l'intégrité et de l'authenticité des données}%https://jwt.io/
}
\newglossaryentry{ldap}{
    name=Lightweight Directory Access Protocoly,
    description={Lightweight Directory Access Protocol (LDAP) est à l'origine un protocole permettant l'interrogation et la modification des services d'annuaire (il est une évolution du protocole DAP)}
}
\newglossaryentry{regex}{
    name=Regex,
    description={En informatique, une expression régulière est une chaîne de caractères qui décrit, selon une syntaxe précise, un ensemble de chaînes de caractères possibles}
}
\newglossaryentry{hash}{
    name=Hash,
    description={On nomme fonction de hachage, de l'anglais hash function (hash : pagaille, désordre, recouper et mélanger) par analogie avec la cuisine, une fonction particulière qui, à partir d'une donnée fournie en entrée, calcule une empreinte numérique servant à identifier rapidement la donnée initiale, au même titre qu'une signature pour identifier une personne}
}
\newglossaryentry{url}{
    name=URL,
    description={Une URL (Uniform Resource Locator), couramment appelée adresse web, est une chaîne de caractères uniforme qui permet d'identifier une ressource du World Wide Web par son emplacement et de préciser le protocole internet pour la récupérer (par exemple http ou https)}
}
\newglossaryentry{websockets}{
    name=Websockets,
    description={WebSocket est un standard du Web désignant un protocole réseau1 de la couche application et une interface de programmation du World Wide Web visant à créer des canaux de communication full-duplex par-dessus une connexion TCP pour les navigateurs web}
}
\newglossaryentry{broadcast}{
    name=Broadcast,
    description={La notion de broadcast est employée par les techniciens en informatique et réseaux ; il s'agit à proprement parler, de transmission ou de liaison. Le principe de base est le même que la télédiffusion, étant donné que l'on diffuse des paquets de données à de nombreux clients éventuellement sans discrimination}
}
\newglossaryentry{repository}{
    name=Repository,
    description={En informatique, un dépôt (de l'anglais repository) est un stockage centralisé et organisé de données}
}
\newglossaryentry{ssh}{
    name=SSH,
    description={Secure Shell (SSH) est à la fois un programme informatique et un protocole de communication sécurisé. Le protocole de connexion impose un échange de clés de chiffrement en début de connexion}
}
\newglossaryentry{vpn}{
    name=VPN,
    description={En informatique, un réseau privé virtuel est un système permettant de créer un lien direct entre des ordinateurs distants, qui isole leurs échanges du reste du trafic se déroulant sur des réseaux de télécommunication publics}
}
\newglossaryentry{dns}{
    name=DNS,
    description={Le Domain Name System ou DNS est un service informatique distribué utilisé qui traduit les noms de domaine Internet en adresse IP ou autres enregistrements}
}
\newglossaryentry{proxy}{
    name=Proxy,
    description={Un proxy est un composant logiciel informatique qui joue le rôle d'intermédiaire en se plaçant entre deux hôtes pour faciliter ou surveiller leurs échanges}
}
\newglossaryentry{https}{
    name=HTTPS,
    description={L'HyperText Transfer Protocol Secure (HTTPS, littéralement « protocole de transfert hypertextuel sécurisé ») est la combinaison du HTTP avec une couche de chiffrement comme SSL ou TLS}
}
\newglossaryentry{tls}{
    name=TLS,
    description={La Transport Layer Security (TLS) ou « Sécurité de la couche de transport », et son prédécesseur la Secure Sockets Layer (SSL) ou « Couche de sockets sécurisée », sont des protocoles de sécurisation des échanges par réseau informatique, notamment par Internet}
}
\newglossaryentry{spa}{
    name=SPA,
    description={Une single-page application (SPA) est une application web ou un site web qui interagit avec l'utilisateur en réécrivant dynamiquement la page web actuelle avec de nouvelles données provenant du serveur web, au lieu de la méthode par défaut du navigateur web qui charge de nouvelles pages entières}
}