\newglossaryentry{heig-vd}{
    name=HEIG-VD,
    description={Haute École d'Ingénierie et de Gestion du canton de Vaud}
}
\newglossaryentry{hes-so}{
    name=HES-SO,
    description={Haute École Supérieure de Suisse Occidentale}
}
\newglossaryentry{latex}{
    name=latex,
    description={Un langage et un système de composition de documents}
}
\newglossaryentry{maths}{
    name=mathematics,
    description={Les mathematiques sont ce que les mathématiciens fonts}
}
\newglossaryentry{docker}{
    name=Docker,
    description={Docker est un outil permettant de gérer des containers, une sorte de machine virtuelle plus légère ayant pour but d'encapsuler une ou plusieurs applications / outils technologiques ainsi que toutes les dépendances que ces dernières exigent pour leur bon fonctionnement}
}
\newglossaryentry{kubernetes}{
    name=Kubernetes,
    description={Kubernetes est un système open-source permettant d'automatiser le déploiement, la mise à l'échelle et la gestion des applications conteneurisées}
}
\newglossaryentry{devops}{
    name=DevOps,
    description={Défini tout le processus conteant l'intégration continue, le déploiement continu et la livraison continue}
}
\newglossaryentry{ci}{
    name=CI,
    description={Continuous Integration / Intégration continue}
}
\newglossaryentry{cd}{
    name=CD,
    description={Continuous Deployement / Déploiement continu}
}
\newglossaryentry{github}{
    name=Github,
    description={GitHub is a website and cloud-based service that helps developers store and manage their code, as well as track and control changes to their code}
}
\newglossaryentry{gitlab}{
    name=Gitlab,
    description={GitLab is The DevOps Platform, delivered as a single application. This makes GitLab unique and creates a streamlined software workflow, unlocking your organization from the constraints of a pieced together toolchain}
}
\newglossaryentry{sgbd}{
    name=SGBD,
    description={Système de Gestion de Base de Données (DBMS), logiciel système servant à stocker, à manipuler ou gérer, et à partager des données dans une base de données}
}
\newglossaryentry{mysql}{
    name=MySQL,
    description={Système de gestion de bases de données relationnelles (SGBDR)}
}
\newglossaryentry{fablab}{
    name=Fablab,
    description={Laboratoire permettant tout type de travaux situé à la HEIG-VD }
}
\newglossaryentry{teams}{
    name=Microsoft Teams,
    description={Plateforme collaborative de visioconférence appartenant à Microsoft}
}
\newglossaryentry{backend}{
    name=Backend,
    description={Terme désignant un étage de sortie d'un logiciel devant produire un résultat. On l'oppose au front-end (aussi appelé un frontal) qui lui est la partie visible de l'iceberg}
}
\newglossaryentry{frontend}{
    name=Frontend,
    description={Partie frontale du projet affichant l'interface utilisateur permettant d'utiliser la logique du backen}
}
\newglossaryentry{bd}{
    name=BD,
    description={Une base de données permet de stocker et de retrouver des données structurées, semi-structurées ou des données brutes ou de l'information, souvent en rapport avec un thème ou une activité. Celles-ci peuvent être de natures différentes et plus ou moins reliées entre elles}
}
\newglossaryentry{rest}{
    name=REST,
    description={REST (representational state transfer) est un style d'architecture logicielle définissant un ensemble de contraintes à utiliser pour créer des services web}
}
\newglossaryentry{api}{
    name=API,
    description={Une interface de programmation d’applications ou interface de programmation applicative (souvent désignée par le terme API pour Application Programming Interface) est un ensemble normalisé de classes, de méthodes, de fonctions et de constantes qui sert de façade par laquelle un logiciel offre des services à d'autres logiciels}
}
\newglossaryentry{framework}{
    name=Framework,
    description={En programmation informatique, un framework est un ensemble cohérent de composants logiciels structurels qui sert à créer les fondations ainsi que les grandes lignes de tout ou partie d'un logiciel, c'est-à-dire une architecture}
}
\newglossaryentry{php}{
    name=PHP,
    description={Language de programmation web}
}
\newglossaryentry{laravel}{
    name=Laravel,
    description={Framework web utilisant le language de programmation php}
}
\newglossaryentry{template}{
    name=Template,
    description={Modèle de projet web frontend définissant déjà toute l'interface graphique et étant adaptable au besoin}
}
\newglossaryentry{javascript}{
    name=JavaScript,
    description={Language de programmation web}
}
\newglossaryentry{vuejs}{
    name=Vue.js,
    description={Framework web frontend utilisant le language de programmation Javascript}
}
\newglossaryentry{nodejs}{
    name=Node.js,
    description={Environnement d’exécution JavaScript construit sur le moteur JavaScript V8}
}
\newglossaryentry{ide}{
    name=IDE,
    description={Un environnement de développement est un ensemble d'outils qui permet d'augmenter la productivité des programmeurs qui développent des logiciels}
}
\newglossaryentry{os}{
    name=OS,
    description={En informatique, un système d'exploitation (souvent appelé OS — de l'anglais Operating System) est un ensemble de programmes qui dirige l'utilisation des ressources d'un ordinateur par des logiciels applicatifs}
}
\newglossaryentry{conteneur}{
    name=conteneur,
    description={Un conteneur est une sorte de machine virtuelle allégée contenant une application}
}
\newglossaryentry{tcp}{
    name=TCP,
    description={Transmission Control Protocol, abrégé TCP, est un protocole de transport fiable, en mode connecté, documenté dans la RFC 7931 de l’IETF}
}
\newglossaryentry{git}{
    name=Git,
    description={}
}
\newglossaryentry{apache}{
    name=Apache,
    description={}
}
\newglossaryentry{orm}{
    name=ORM,
    description={}
}
\newglossaryentry{eloquent}{
    name=Eloquent,
    description={}
}
\newglossaryentry{json}{
    name=JSON,
    description={}
}
\newglossaryentry{http}{
    name=HTTP,
    description={}
}
\newglossaryentry{jwt}{
    name=JWT,
    description={}%https://jwt.io/
}
\newglossaryentry{ad}{
    name=Active Directory,
    description={}
}
\newglossaryentry{regex}{
    name=Regex,
    description={}
}
\newglossaryentry{hash}{
    name=Hash,
    description={}
}
\newglossaryentry{url}{
    name=URL,
    description={}
}
\newglossaryentry{websockets}{
    name=Websockets,
    description={}
}
\newglossaryentry{broadcast}{
    name=Broadcast,
    description={}
}
\newglossaryentry{repository}{
    name=Repository,
    description={}
}
\newglossaryentry{ssh}{
    name=SSH,
    description={}
}
\newglossaryentry{vpn}{
    name=VPN,
    description={}
}
\newglossaryentry{dns}{
    name=DNS,
    description={}
}
\newglossaryentry{proxy}{
    name=Proxy,
    description={}
}
\newglossaryentry{https}{
    name=HTTPS,
    description={}
}
\newglossaryentry{spa}{
    name=SPA,
    description={}
}