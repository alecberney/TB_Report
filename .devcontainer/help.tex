\chapter{Aides}

\section{Exemple d'équation}
L'une des principales forces de \LaTeX est la saisie d'équations. L'équation \ref{eq:1}, citée à titre d'exemple, représente la transformation de phase d'une lentille biconvexe. Pour rédiger une équation \LaTeX vous pouvez utiliser des outils en ligne tels que \href{https://www.latex4technics.com/}{latex4technics}.

\begin{equation} \label{eq:1}
    \begin{split}
        L(x,y) &= \exp\left( - i\frac{{2\pi }}{\lambda }\left( {n\Delta \varphi (x,y) + \Delta {\varphi _0} - \Delta \varphi (x,y)} \right)\right)\\
        &= {\exp\left({i\frac{{2\pi }}{\lambda }\Delta {\varphi _0}}\right)}{\exp\left({ - i\frac{{2\pi }}{{\lambda f}}({x^2} + {y^2})}\right)}
    \end{split}
\end{equation}

\section{Exemples de diagrammes}

Les diagrammes de flux peuvent être réalisés en utilisant l'outil \href{https://app.diagrams.net/}{draw.io}. Une exportation en \texttt{.xml} (non compressé) permet de garder les sources de la figure. Le rendu en \texttt{.pdf} sera réalisé à la volée à la compilation. L'intérêt est double : n'avoir qu'une source de vérité c'est à dire pas d'image intermédiaire à stocker, et réduire la quantité d'information stockée.

Puisque la source est au format XML, les textes sont accessibles au correcteur orthographique et il vous est rendu possible les modifier sans avoir à éditer l'image. La figure \ref{euclide.xml} en est un exemple.


\figi{euclide.xml}{9cm}{Algorithme d'Euclide}

Notons qu'il est inutile d'insérer des images coloriées là où la couleur n'offre aucune valeur ajoutée ; évitez également les ombrages et autres effets de style. Enfin, préférez toujours des représentations vectorielles là où c'est possible.

Voici un autre type de diagramme utile (figure \ref{sequence.xml}), celui d'une séquence UML.

\figi{sequence.xml}{8cm}{Diagramme de séquence}

\section{Exemple de figure}

Pour présenter des résultats d'expérience, vous pouvez soit dessiner des graphiques manuellement en utilisant des outils de dessin vectoriel comme Inkscape ou Adobe Illustrator comme illustré à la figure \ref{plot.svg} ou alors, vous pouvez utiliser Python ou Matlab. Avec ce dernier choix, vous pouvez générer vos figures à la volée : le code source \ref{python} permet de générer la figure \ref{bode.py}.

\fig{plot.svg}{Exemple de graphique plan}

\begin{listing}[h]
    \inputminted{python}{assets/figures/bode.py}
    \caption{Génération d'un diagramme de Bode \label{python}}
\end{listing}

\figi{bode.py}{12cm}{Diagramme de Bode généré à la volée}

\clearpage

\subsection{Schémas électroniques}
Vous pouvez également utiliser TikZ pour créer vos propres schémas électriques et électroniques comme l'exemple \ref{circuit}.

\begin{figure}[h]
    \begin{center}
        \begin{circuitikz}
            \draw
            (0,0) to [short, *-] (6,0)
            to [V, l_=$\mathrm{j}{\omega}_m \underline{\phi}^s_R$] (6,2)
            to [R, l_=$R_R$] (6,4)
            to [short, i_=$\underline{i}^s_R$] (5,4)
            (0,0) to [open, v^>=$\underline{u}^s_s$] (0,4)
            to [short, *- ,i=$\underline{i}^s_s$] (1,4)
            to [R, l=$R_s$] (3,4)
            to [L, l=$L_{\sigma}$] (5,4)
            to [short, i_=$\underline{i}^s_M$] (5,3)
            to [L, l_=$L_M$] (5,0);
        \end{circuitikz}
        \caption{Circuit électrique \label{circuit}}
    \end{center}
\end{figure}

\subsection{Dessins techniques}
L'intégration de dessins mécaniques est préférée en vue filaire. SolidWorks conserve la représentation vectorielle à l'exportation. À partir du PDF généré, l'image peut être isolée et sauvegardée en format SVG.

\begin{figure}[!ht]
    \begin{center}
        \includegraphics[width=10cm]{\assetsdir/assembly.svg.\graphicsExt}
    \end{center}
    \caption[Assemblage mécanique]{\label{assembly}Réducteur cycloïdale de puissance comportant 6. l'axe de sortie, 14. le roulement de sortie, 1. le corps du réducteur en aluminium, 3 et 5. les disques cycloïdaux et 2. les goupilles de prise... D'autres informations liées à la figure elle-même peuvent aussi figurer dans la légende}
\end{figure}

Notez ici que la légende est particulièrement longue. Celle que vous retrouverez dans la table figures est plus courte. La commande \mintinline{latex}{\caption[courte]{longue}} permet de saisir une légende courte, pour la table des figures et longue pour le corps du document.

La figure \ref{assembly} est un dessin technique épuré qui permet de décrire un phénomène ou un fonctionnement important dans le rapport technique. Les mises en plan détaillées seront quant à elles disponibles en annexes.

\clearpage
\section{Tableaux}

Concernant les tableaux, restez simple et minimaliste, n'ajoutez des séparateurs que là ou c'est nécessaire pour améliorer la lisibilité. Une liste de quelques cantons suisses est donnée à titre d'exemple dans la table \ref{cantons}.

\begin{table}[h]
    \begin{center}
        \caption{Liste des cantons \label{cantons}}
        \begin{tabular}{c|l|r}
            Abréviation & Nom du canton & Depuis                  \\ \hline
            ZH          & Zürich        & \ordinalnum{1} mai 1351 \\
            BE          & Berne         & 6 mars 1353             \\
            FR          & Fribourg      & 22 décembre 1481        \\
            VD          & Vaud          & 19 février 1815         \\
            VS          & Valais        & 4 août 1815             \\
            NE          & Neuchâtel     & 19 mai 1815             \\
            GE          & Genève        & 19 mai 1815
        \end{tabular}
    \end{center}
\end{table}

Si vous devez donner une spécification technique, n'oubliez pas de mentionner les valeurs minimales, maximales et nominales sans omettre l'unité de mesure. Notez que les séparateurs verticaux sont souvent critiqués pour réduire la lisibilité mais parfois ils sont utiles. Utilisez-les avec parcimonie.

\begin{table}[h]
    \begin{center}
        \caption{Exigences techniques \label{specification}}
        \begin{tabularx}{\textwidth}{cXcccc}
            No. & Exigence                                                                   & Min. & Nom. & Max. & Unité                           \\ \toprule
            E1  & Tension d'alimentation                                                     & 12   & 24   & 48   & \si{\volt}                      \\ \midrule
            E2  & Fréquence                                                                  & 50   &      & 60   & \si{\hertz}                     \\ \midrule
            E3  & Concentration                                                              &      & 300  & 1200 & \si{\nano\gram\per\milli\litre} \\ \midrule
            E4  & \multicolumn{5}{l}{Doit pouvoir être stoppé à l'aide d'un arrêt d'urgence}
        \end{tabularx}
    \end{center}
\end{table}

L'exemple de la table \ref{specification}, assigne pour chaque exigence un numéro unique. Cette table est \textbf{normative}, chaque élément doit pouvoir être référencé par un identifiant unique (cf. T\ref{specification}-E3). Dans le cas ou cet identifiant est utilisé en dehors de ce document, la version du document devra être renseignée.

\section{Index}
\LaTeX~ permet d'indexer les mots \index{mots} importants. Il suffit de placer les termes importants d'un paragraphe dans la commande \texttt{\textbackslash index\{terme\}} et ils apparaîtront automatiquement à la fin de ce rapport dans l'index du document.

\index{Napoléon}

Imaginons que dans cette section nous parlions du cheval blanc \index{cheval blanc} de Napoléon. Il se pourrait que le lecteur recherche ce passage dans la version imprimée du rapport. Avec l'index, rien de plus facile. Allez jeter un oeil à la page \pageref{index}.

\section{Notes de bas de page}

\maraja{Je suis une marginale, et je suis utile pour résumé un paragraphe en quelques mots.} Parfois, il est plus élégant d'annoter une définition en utilisant une note de bas de page \footnote{La note en bas de page (ou note de bas de page) est une forme littéraire, consistant en une ou plusieurs lignes ne figurant pas dans le texte.}. Alternativement il est possible d'annoter un paragraphe avec une note marginale.

\section{Glossaire et acronymes}

La \Gls{heig-vd} membre de la \Gls{hes-so} propose ce modèle de document. Le format \LaTeX est particulièrement adapté pour les documents qui contiennent des expressions mathématiques. Pour plus de détail sur l'utilisation d'un glossaire, se référer à \url{https://www.overleaf.com/learn/latex/Glossaries}. Tient donc, ci-dessus nous utilisons deux acronymes. Les trouverez-vous dans le glossaire en page \pageref{glossaire} ?

\section{Unités de mesure}

Lorsque vous mentionnez des quantités, utilisez les unités du système international. \LaTeX~et le paquet \textsf{siunitx} permet la saisie de quantités. La commande suivante permet d'afficher \SI{42.12}{\kilo\gram\metre\per\square\second}.\par
