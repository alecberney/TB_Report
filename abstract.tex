% Francais
Le fablab est un laboratoire de l'HEIG-VD où l'on peut réaliser toutes sortes de travaux à l'aide de machines ou autres. Ces travaux sont en générales demandés par des élèves auprès de techniciens. La gestion de ces demandes ne convient pas aux responsables du fablab et une plateforme web dédiée pour gérer ces demandes a déjà fait l'objet d'un premier jet lors d'un précédent travail de Bachelor. Ce dernier n'étant pas tout à fait terminer et déployer, un nouveau travail de Bachelor a été proposé afin de l'améliorer et de le terminer.
La plateforme web fournie permettra ainsi une meilleure gestion des échanges et commandes.
Cela aménera également une nouvelle dimension au niveau du traçage et l'administration du laboratoire.
Une procédure claire ayant de la demande à la réalisation du travail demandé sera également mise en place grâce à cette application.
Pour faciliter le suivi des demandes client, la plateforme mettra en place un système de notifications par email et sur l'application même.
L'outil web ayant pour but de toucher les utilisateurs de l'école, l'authentification via l'outil Switch edu-id sera fourni.

\asterism

% English
The fablab is a laboratory of the HEIG-VD school where mainly students can realize all kinds of works with the help of machines / devices or others. These works are generally requested by students to technicians / experts. The management of these requests does not suit the fablab managers and a dedicated web platform to manage these requests has already been the subject of a previous Bachelor work. As the application is not totally finished and deployed, a fresh Bachelor work has been proposed in order to improve and complete it.
The final web platform will allow a better management of exchanges and orders.
It will also provide a new dimension to the tracking and administration of the laboratory.
A clear procedure from the request to the realization of the requested work will also be implemented through this application.
To facilitate the follow-up of customer requests, the platform will set up a notification system by email and on the application itself.
As the web tool is intended to reach school users, authentication via the Switch edu-id tool will be provided.